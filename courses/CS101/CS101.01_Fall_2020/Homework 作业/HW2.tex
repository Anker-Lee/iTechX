\documentclass[10.5pt]{article}
\usepackage{amsmath,amssymb,amsthm}
\usepackage{listings}
\usepackage{graphicx}
\usepackage[shortlabels]{enumitem}
\usepackage{tikz}
\usepackage[margin=1in]{geometry}
\usepackage{fancyhdr}
\usepackage{epsfig} %% for loading postscript figures
\usepackage{amsmath}
\usepackage{float}
\usepackage{amssymb}
\usepackage{caption}
\usepackage{subfigure}
\usepackage{graphics}
\usepackage{titlesec}
\usepackage{mathrsfs}
\usepackage{amsfonts}
\usepackage{indentfirst}
\usepackage{fancybox}
\usepackage{tikz}
\usepackage{algorithm}
\usepackage{algcompatible}
% \usepackage{fontspec}

\renewcommand{\baselinestretch}{1.2}%Adjust Line Spacing
%\geometry{left=2.0cm,right=2.0cm,top=2.0cm,bottom=2.0cm}% Adjust Margins of the File
\usepackage{tikz-qtree}
\usetikzlibrary{graphs}
\tikzset{every tree node/.style={minimum width=2em,draw,circle},
	blank/.style={draw=none},
	edge from parent/.style=
	{draw,edge from parent path={(\tikzparentnode) -- (\tikzchildnode)}},
	level distance=1.2cm}
\setlength{\parindent}{0pt}
%\setlength{\parskip}{5pt plus 1pt}
\setlength{\headheight}{13.6pt}
\newcommand\question[2]{\vspace{.25in}\hrule\textbf{#1: #2}\vspace{.5em}\hrule\vspace{.10in}}
\renewcommand\part[1]{\vspace{.10in}\textbf{(#1)}}
%\newcommand\algorithm{\vspace{.10in}\textbf{Algorithm: }}
\newcommand\correctness{\vspace{.10in}\textbf{Correctness: }}
\newcommand\runtime{\vspace{.10in}\textbf{Running time: }}
\pagestyle{fancyplain}
% Create horizontal rule command with an argument of height
\newcommand{\horrule}[1]{\rule{\linewidth}{#1}}
% Set the title here
\title{
	\normalfont \normalsize
	\textsc{ShanghaiTech University} \\ [25pt]
	\horrule{0.5pt} \\[0.4cm] % Thin top horizontal rule
	\huge CS101 Algorithms and Data Structures\\ % The assignment title
	\LARGE Fall 2020\\
	\LARGE Homework 2\\
	\horrule{2pt} \\[0.5cm] % Thick bottom horizontal rule
}
% wrong usage of \author, never mind
\author{}
\date{Due date: 23:59, September 21, 2020}

% set the header and footer
\pagestyle{fancy}
\lhead{CS101 Algorithms and Data Structures}
\chead{Homework 2}
\rhead{Due date: 23:59, September 21, 2020}
\cfoot{\thepage}
\renewcommand{\headrulewidth}{0.4pt}
\newtheorem{Q}{Question}
% special settings for the first page
\fancypagestyle{firstpage}
{
	\renewcommand{\headrulewidth}{0pt}
	\fancyhf{}
	\fancyfoot[C]{\thepage}
}

% Add the support for auto numbering
% use \problem{title} or \problem[number]{title} to add a new problem
% also \subproblem is supported, just use it like \subsection
\newcounter{ProblemCounter}
\newcounter{oldvalue}
\newcommand{\problem}[2][-1]{
	\setcounter{oldvalue}{\value{secnumdepth}}
	\setcounter{secnumdepth}{0}
	\ifnum#1>-1
	\setcounter{ProblemCounter}{0}
	\else
	\stepcounter{ProblemCounter}
	\fi
	\section{Problem \arabic{ProblemCounter}: #2}
	\setcounter{secnumdepth}{\value{oldvalue}}
}
\newcommand{\subproblem}[1]{
	\setcounter{oldvalue}{\value{section}}
	\setcounter{section}{\value{ProblemCounter}}
	\subsection{#1}
	\setcounter{section}{\value{oldvalue}}
}

% \setmonofont{Consolas}
\definecolor{blve}{rgb}{0.3372549 , 0.61176471, 0.83921569}
\definecolor{gr33n}{rgb}{0.29019608, 0.7372549 , 0.64705882}
\makeatletter
\lst@InstallKeywords k{class}{classstyle}\slshape{classstyle}{}ld
\makeatother
\lstset{language=C++,
	basicstyle=\ttfamily,
	keywordstyle=\color{blve}\ttfamily,
	stringstyle=\color{red}\ttfamily,
	commentstyle=\color{green}\ttfamily,
	morecomment=[l][\color{magenta}]{\#},
	classstyle = \bfseries\color{gr33n}, 
	tabsize=4
}

\begin{document}
	\maketitle
	\thispagestyle{firstpage}
	%\newpage
	\vspace{3ex}
	
	\begin{enumerate}
		\item Please write your solutions in English. 
		
		\item Submit your solutions to gradescope.com.  
		
		\item Set your Full Name to your Chinese name and your STUDENT ID correctly in Account Settings. 
		
		\item If you want to submit a handwritten version, scan it clearly. Camscanner is recommended. 
		
		\item When submitting, match your solutions to the according problem numbers correctly. 
		
		\item No late submission will be accepted.
		
		\item Violations to any of above may result in zero score. 
	\end{enumerate}
	\newpage
	
	%---------------------------------------------------------
\question{1}{(4*1') Multiple Choices}

Each question has one or more correct answer(s). Select all the correct answer(s). For each question, you get $0$ point if you select one or more wrong answers, but you get $0.5$ point if you select a non-empty subset of the correct answers.\\
\textit{Note that you should write you answers of section 1 in the table below.}
\begin{table}[htbp]
	\begin{tabular}{|p{2cm}|p{2cm}|p{2cm}|p{2cm}|}
		\hline 
		Question 1 & Question 2 & Question 3 & Question 4  \\ 
		\hline 
		&  &  &  \\ 
		\hline 
	\end{tabular} 
\end{table}

\begin{Q}
	Using linked list to implement a stack, where are the pushes and pops performed?
	\begin{enumerate}[(A)]
		\item Push in front of the first element, pop the first element
		\item Push after the last element, pop the last element
		\item Push after the last element, pop the first element
		\item Push in front of the first element, pop the last element
		\item Push after the first element, pop the first element
	\end{enumerate}
\end{Q}

\begin{Q}
	Suppose we use a circular array with index range from 0 to N - 1 to implement a queue, and currently Front is pointing to index $m$. We will know that this queue is full if \textbf{Back} is point to index = \rule[-3pt]{1cm}{0.05em}. (Options below are \textbf{Integers Modulo N}: $\mathbb{Z}_N = \{0, 1, 2, ..., N-1\}$)
	\begin{enumerate}[(A)]
		\item $0$
		\item $m$
		\item $N-1$
		\item $m-1$
	\end{enumerate}
\end{Q}

\begin{Q}
	Stack cannot be used for\rule[-3pt]{2cm}{0.05em}
	\begin{enumerate}[(A)]
		\item IO buffers
		\item resource allocation and scheduling
		\item compilers/word processors
		\item handling function calls
	\end{enumerate}
\end{Q}

\begin{Q}
	What function does the following code achieve?
	\rm{
\begin{lstlisting}[language=C++]
  void Q4(Queue &Q) 
  {
    Stack S;
	int d;
	S.InitStack();
	while(!Q.IsEmpty())
	{
	  d = Q.DeQueue();
	  S.Push(d);
	}
	while(!S.IsEmpty())
	{
	  d = S.Pop();
	  Q.EnQueue(d);
	}
  }
\end{lstlisting}
	}
	\begin{enumerate}[(A)]
		\item Use a stack to reverse a queue.
		\item Use a queue to reverse a stack.
		\item Use a stack to implement a queue.
		\item Use a queue to implement a stack.
	\end{enumerate}
\end{Q}

	%---------------------------------------------------------
\question{2}{(3'+8') Stack and Queue}

\begin{Q}
\textbf{(1.5'+1.5')}\\Consider the following sequence of stack operations: 
\center{\textbf{push(d), push(h), pop(), push(f), push(s), pop(), pop(), push(m).}}
\begin{enumerate}[(a)]
	\item We assume the stack is initially empty,  please write down the sequence of the popped values, and the final stack (Please label in your answer the top and the bottom of the stack) .
	\item Now we work on a queue instead, the queue goes through the same operations as listed above. What will be the sequence of the popped values, and the final queue? (Please label in your answer the front and the back of the queue)
\end{enumerate}
\vspace{2.5in}
\end{Q}

\pagebreak
%--------------------------------------------------------------
	\begin{Q}\textbf{(8')Postfix expression}\\
	Reverse Polish notation (RPN) is a mathematical notation in which operators follow their operands. Using a stack, we can evaluate postfix notation equations easily.
	
	\begin{enumerate}[(a)]
		\item \textbf{Calculating(1'+2')}
		
		A post-fix expression (Reverse-Polish Notation) with single digit operands is shown below:\\
		$$\mathtt{8\ 2\ 3\ -\ /\ 2\ 3\ *\ +\ 5\ 1\ *\ -}$$
		
		
		Its in-fix expression is: \_\_\_\_\_\_\_\_\_\_\_\_\_\_\_\_\_\_\_\_\_\_\_\_\_\_\_\_\_\_\_\_\\
		The changing of the stack to calculate the final result is:
		\begin{table}[htb]
			\centering\begin{tabular}{|l|l|l|l|l|l|l|l|l|l|l|l|l|l|l|l|l|l|l|l|l|l|l|}
				\cline{1-1} \cline{3-3} \cline{5-5} \cline{7-7} \cline{9-9} \cline{11-11} \cline{13-13} \cline{15-15} \cline{17-17} \cline{19-19} \cline{21-21} \cline{23-23}
				&  &  &  &  &  &  &  &  &  &  &  &  &  &  &  &  &  &  &  &  &  &  \\ \cline{1-1} \cline{3-3} \cline{5-5} \cline{7-7} \cline{9-9} \cline{11-11} \cline{13-13} \cline{15-15} \cline{17-17} \cline{19-19} \cline{21-21} \cline{23-23} 
				&  &  &  &  &  &  &  &  &  &  &  &  &  &  &  &  &  &  &  &  &  &  \\ \cline{1-1} \cline{3-3} \cline{5-5} \cline{7-7} \cline{9-9} \cline{11-11} \cline{13-13} \cline{15-15} \cline{17-17} \cline{19-19} \cline{21-21} \cline{23-23} 
				&  &  &  &  &  &  &  &  &  &  &  &  &  &  &  &  &  &  &  &  &  &  \\ \cline{1-1} \cline{3-3} \cline{5-5} \cline{7-7} \cline{9-9} \cline{11-11} \cline{13-13} \cline{15-15} \cline{17-17} \cline{19-19} \cline{21-21} \cline{23-23} 
				8 &  &  &  &  &  &  &  &  &  &  &  &  &  &  &  &  &  &  &  &  &  &  \\ \cline{1-1} \cline{3-3} \cline{5-5} \cline{7-7} \cline{9-9} \cline{11-11} \cline{13-13} \cline{15-15} \cline{17-17} \cline{19-19} \cline{21-21} \cline{23-23} 
			\end{tabular}
		\end{table}

		\item \textbf{Conversion(2*0.5'+2*1')}
		Now, try to convert the following in-fix expression into post-fix expression: (You don't need to calculate them)\\
		Note that \^\ \ represents the exponentiation operator. 
		\begin{enumerate}[1) ]
			\item $ 1+2+3 $\\
			\\
			\item $ 1+5*9 $\\
			\\
			\item $ 1 + 3*5 + (2 * 4 + 6) * 8 $\\
			\\
			\item $ 5 + 8*9  \hat\   2 /(5+3)+1$\\
			\\
		\end{enumerate}
		\item \textbf{Validity(4*0.5')}\\
		Please judge whether the following post-fix expression is legal, if legal, please write \textbf{T}, otherwise please write \textbf{F}.
		\begin{enumerate}
			\item $ 1\ 2\ +\ -\ 3\ 5\ +\ $
			\item $ 4\ 5\ 6\ /\ *\ 1\ /\ $
			\item $ 1\ +\ 2\ -\ 3\ +\ 4\ $
			\item $ 7\ 8\ 9\ 1\ +\ -\ *\ $
		\end{enumerate}
	\end{enumerate}
	\end{Q}
	\pagebreak
	

\question{3}{(5'+5') Big-O Notation}
\begin{Q}
 \textbf{(5')} \textbf{Asymptotic Analysis}
 
For each pair of functions $f(n)$ and $g(n)$, give your answer whether $f(n) = O(g(n))$, $f(n) = \Omega(g(n))$ or $f(n) = \Theta(g(n))$.  (Try to give your answer in the most precise form.) For example, for $f(n) = n^2$ and $g(n) = n^2 + 2n + 1$, write $f(n) = \Theta(g(n))$. \textbf{Briefly} justify your answers.
\begin{enumerate}
    \item $f(n) = n^{100}$ and $g(n) = {1.01}^n$
    \\
    \\
    \\
    \\
    \item $f(n) = n^{1.01}$ and $g(n) = n\log n$
    \\
    \\
    \\
    \\
    \item $f(n) = 3n + \log n$ and $g(n) = 2n + {\log}^2 n$ 
    \\
    \\
    \\
    \\
    \item $f(n) = n!$ and $g(n) = n^n$
    \\
    \\
    \\
    \\
    \item $f(n) = {\log}^2(n)$ and $g(n) = \log \log n$ 
    \\
    \\
    \\
    \\
\end{enumerate}
\end{Q}
\pagebreak
\begin{Q}
\textbf{(5')} \textbf{Complexity Analysis}
\\
Consider the factorial computing problem: $N! = 1 \times 2 \times \cdots \times N $.
\begin{enumerate}
    \item (2') To store the number $N$, we need at least $\log N$ bits of space. Find an $f(N)$ such that $N!$ is $\Theta(f(N))$ bits long. Simplify your answer and justify your answer.
    \\
    \\
    \\
    \\
    \\
    \\
    \\
    \item (3') Consider this naive factorial computing algorithm.
        \algnewcommand\algorithmicreturn{\textbf{return}}
        \algnewcommand\RETURN{\State \algorithmicreturn}%
        \algnewcommand\algorithmicprocedure{\textbf{procedure}}
        \algnewcommand\PROCEDURE{\item[\algorithmicprocedure]}%
        \algnewcommand\algorithmicendprocedure{\textbf{end procedure}}
        \algnewcommand\ENDPROCEDURE{\item[\algorithmicendprocedure]}%
        \begin{algorithmic}
        \PROCEDURE \algvar{FACTORIAL}(\algarg{$N$})
        \STATE $f = 1$
        \FOR{$i = 2$ to $N$}
        \STATE $f = f \cdot i$
        \ENDFOR
        \RETURN{} $f$
        \ENDPROCEDURE
        \end{algorithmic}
    We assume the runtime of multiplying an m-bit number and an n-bit number is $\Theta(mn)$.
    What's the runtime of this factorial computing algorithm in Big-O notation? Justify your answers.
        

\end{enumerate}

\end{Q}


\end{document}